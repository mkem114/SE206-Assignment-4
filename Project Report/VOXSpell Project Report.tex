
\documentclass[conference]{IEEEtran}
% Some Computer Society conferences also require the compsoc mode option,
% but others use the standard conference format.
%
% If IEEEtran.cls has not been installed into the LaTeX system files,
% manually specify the path to it like:
% \documentclass[conference]{../sty/IEEEtran}





% Some very useful LaTeX packages include:
% (uncomment the ones you want to load)


% *** MISC UTILITY PACKAGES ***
%
%\usepackage{ifpdf}
% Heiko Oberdiek's ifpdf.sty is very useful if you need conditional
% compilation based on whether the output is pdf or dvi.
% usage:
% \ifpdf
%   % pdf code
% \else
%   % dvi code
% \fi
% The latest version of ifpdf.sty can be obtained from:
% http://www.ctan.org/pkg/ifpdf
% Also, note that IEEEtran.cls V1.7 and later provides a builtin
% \ifCLASSINFOpdf conditional that works the same way.
% When switching from latex to pdflatex and vice-versa, the compiler may
% have to be run twice to clear warning/error messages.






% *** CITATION PACKAGES ***
%
%\usepackage{cite}
% cite.sty was written by Donald Arseneau
% V1.6 and later of IEEEtran pre-defines the format of the cite.sty package
% \cite{} output to follow that of the IEEE. Loading the cite package will
% result in citation numbers being automatically sorted and properly
% "compressed/ranged". e.g., [1], [9], [2], [7], [5], [6] without using
% cite.sty will become [1], [2], [5]--[7], [9] using cite.sty. cite.sty's
% \cite will automatically add leading space, if needed. Use cite.sty's
% noadjust option (cite.sty V3.8 and later) if you want to turn this off
% such as if a citation ever needs to be enclosed in parenthesis.
% cite.sty is already installed on most LaTeX systems. Be sure and use
% version 5.0 (2009-03-20) and later if using hyperref.sty.
% The latest version can be obtained at:
% http://www.ctan.org/pkg/cite
% The documentation is contained in the cite.sty file itself.






% *** GRAPHICS RELATED PACKAGES ***
%
\ifCLASSINFOpdf
% \usepackage[pdftex]{graphicx}
% declare the path(s) where your graphic files are
% \graphicspath{{../pdf/}{../jpeg/}}
% and their extensions so you won't have to specify these with
% every instance of \includegraphics
% \DeclareGraphicsExtensions{.pdf,.jpeg,.png}
\else
% or other class option (dvipsone, dvipdf, if not using dvips). graphicx
% will default to the driver specified in the system graphics.cfg if no
% driver is specified.
% \usepackage[dvips]{graphicx}
% declare the path(s) where your graphic files are
% \graphicspath{{../eps/}}
% and their extensions so you won't have to specify these with
% every instance of \includegraphics
% \DeclareGraphicsExtensions{.eps}
\fi
% graphicx was written by David Carlisle and Sebastian Rahtz. It is
% required if you want graphics, photos, etc. graphicx.sty is already
% installed on most LaTeX systems. The latest version and documentation
% can be obtained at: 
% http://www.ctan.org/pkg/graphicx
% Another good source of documentation is "Using Imported Graphics in
% LaTeX2e" by Keith Reckdahl which can be found at:
% http://www.ctan.org/pkg/epslatex
%
% latex, and pdflatex in dvi mode, support graphics in encapsulated
% postscript (.eps) format. pdflatex in pdf mode supports graphics
% in .pdf, .jpeg, .png and .mps (metapost) formats. Users should ensure
% that all non-photo figures use a vector format (.eps, .pdf, .mps) and
% not a bitmapped formats (.jpeg, .png). The IEEE frowns on bitmapped formats
% which can result in "jaggedy"/blurry rendering of lines and letters as
% well as large increases in file sizes.
%
% You can find documentation about the pdfTeX application at:
% http://www.tug.org/applications/pdftex





% *** MATH PACKAGES ***
%
%\usepackage{amsmath}
% A popular package from the American Mathematical Society that provides
% many useful and powerful commands for dealing with mathematics.
%
% Note that the amsmath package sets \interdisplaylinepenalty to 10000
% thus preventing page breaks from occurring within multiline equations. Use:
%\interdisplaylinepenalty=2500
% after loading amsmath to restore such page breaks as IEEEtran.cls normally
% does. amsmath.sty is already installed on most LaTeX systems. The latest
% version and documentation can be obtained at:
% http://www.ctan.org/pkg/amsmath





% *** SPECIALIZED LIST PACKAGES ***
%
%\usepackage{algorithmic}
% algorithmic.sty was written by Peter Williams and Rogerio Brito.
% This package provides an algorithmic environment fo describing algorithms.
% You can use the algorithmic environment in-text or within a figure
% environment to provide for a floating algorithm. Do NOT use the algorithm
% floating environment provided by algorithm.sty (by the same authors) or
% algorithm2e.sty (by Christophe Fiorio) as the IEEE does not use dedicated
% algorithm float types and packages that provide these will not provide
% correct IEEE style captions. The latest version and documentation of
% algorithmic.sty can be obtained at:
% http://www.ctan.org/pkg/algorithms
% Also of interest may be the (relatively newer and more customizable)
% algorithmicx.sty package by Szasz Janos:
% http://www.ctan.org/pkg/algorithmicx




% *** ALIGNMENT PACKAGES ***
%
%\usepackage{array}
% Frank Mittelbach's and David Carlisle's array.sty patches and improves
% the standard LaTeX2e array and tabular environments to provide better
% appearance and additional user controls. As the default LaTeX2e table
% generation code is lacking to the point of almost being broken with
% respect to the quality of the end results, all users are strongly
% advised to use an enhanced (at the very least that provided by array.sty)
% set of table tools. array.sty is already installed on most systems. The
% latest version and documentation can be obtained at:
% http://www.ctan.org/pkg/array


% IEEEtran contains the IEEEeqnarray family of commands that can be used to
% generate multiline equations as well as matrices, tables, etc., of high
% quality.




% *** SUBFIGURE PACKAGES ***
%\ifCLASSOPTIONcompsoc
%  \usepackage[caption=false,font=normalsize,labelfont=sf,textfont=sf]{subfig}
%\else
%  \usepackage[caption=false,font=footnotesize]{subfig}
%\fi
% subfig.sty, written by Steven Douglas Cochran, is the modern replacement
% for subfigure.sty, the latter of which is no longer maintained and is
% incompatible with some LaTeX packages including fixltx2e. However,
% subfig.sty requires and automatically loads Axel Sommerfeldt's caption.sty
% which will override IEEEtran.cls' handling of captions and this will result
% in non-IEEE style figure/table captions. To prevent this problem, be sure
% and invoke subfig.sty's "caption=false" package option (available since
% subfig.sty version 1.3, 2005/06/28) as this is will preserve IEEEtran.cls
% handling of captions.
% Note that the Computer Society format requires a larger sans serif font
% than the serif footnote size font used in traditional IEEE formatting
% and thus the need to invoke different subfig.sty package options depending
% on whether compsoc mode has been enabled.
%
% The latest version and documentation of subfig.sty can be obtained at:
% http://www.ctan.org/pkg/subfig




% *** FLOAT PACKAGES ***
%
%\usepackage{fixltx2e}
% fixltx2e, the successor to the earlier fix2col.sty, was written by
% Frank Mittelbach and David Carlisle. This package corrects a few problems
% in the LaTeX2e kernel, the most notable of which is that in current
% LaTeX2e releases, the ordering of single and double column floats is not
% guaranteed to be preserved. Thus, an unpatched LaTeX2e can allow a
% single column figure to be placed prior to an earlier double column
% figure.
% Be aware that LaTeX2e kernels dated 2015 and later have fixltx2e.sty's
% corrections already built into the system in which case a warning will
% be issued if an attempt is made to load fixltx2e.sty as it is no longer
% needed.
% The latest version and documentation can be found at:
% http://www.ctan.org/pkg/fixltx2e


%\usepackage{stfloats}
% stfloats.sty was written by Sigitas Tolusis. This package gives LaTeX2e
% the ability to do double column floats at the bottom of the page as well
% as the top. (e.g., "\begin{figure*}[!b]" is not normally possible in
% LaTeX2e). It also provides a command:
%\fnbelowfloat
% to enable the placement of footnotes below bottom floats (the standard
% LaTeX2e kernel puts them above bottom floats). This is an invasive package
% which rewrites many portions of the LaTeX2e float routines. It may not work
% with other packages that modify the LaTeX2e float routines. The latest
% version and documentation can be obtained at:
% http://www.ctan.org/pkg/stfloats
% Do not use the stfloats baselinefloat ability as the IEEE does not allow
% \baselineskip to stretch. Authors submitting work to the IEEE should note
% that the IEEE rarely uses double column equations and that authors should try
% to avoid such use. Do not be tempted to use the cuted.sty or midfloat.sty
% packages (also by Sigitas Tolusis) as the IEEE does not format its papers in
% such ways.
% Do not attempt to use stfloats with fixltx2e as they are incompatible.
% Instead, use Morten Hogholm'a dblfloatfix which combines the features
% of both fixltx2e and stfloats:
%
% \usepackage{dblfloatfix}
% The latest version can be found at:
% http://www.ctan.org/pkg/dblfloatfix




% *** PDF, URL AND HYPERLINK PACKAGES ***
%
%\usepackage{url}
% url.sty was written by Donald Arseneau. It provides better support for
% handling and breaking URLs. url.sty is already installed on most LaTeX
% systems. The latest version and documentation can be obtained at:
% http://www.ctan.org/pkg/url
% Basically, \url{my_url_here}.




% *** Do not adjust lengths that control margins, column widths, etc. ***
% *** Do not use packages that alter fonts (such as pslatex).         ***
% There should be no need to do such things with IEEEtran.cls V1.6 and later.
% (Unless specifically asked to do so by the journal or conference you plan
% to submit to, of course. )


% correct bad hyphenation here
\hyphenation{op-tical net-works semi-conduc-tor}


\begin{document}
	%
	% paper title
	% Titles are generally capitalized except for words such as a, an, and, as,
	% at, but, by, for, in, nor, of, on, or, the, to and up, which are usually
	% not capitalized unless they are the first or last word of the title.
	% Linebreaks \\ can be used within to get better formatting as desired.
	% Do not put math or special symbols in the title.
	\title{VOXSpell Project Tecnical Report}
	
	
	% author names and affiliations
	% use a multiple column layout for up to three different
	% affiliations
	\author{\IEEEauthorblockN{Michael Kemp}
		\IEEEauthorblockA{Software Engineering\\Department of Electrical and Computer Engineering\\
			University of Auckland\\
			Auckland, New Zealand 1010\\
			Email: mkem114@aucklanduni.ac.nz}}
	
	% conference papers do not typically use \thanks and this command
	% is locked out in conference mode. If really needed, such as for
	% the acknowledgment of grants, issue a \IEEEoverridecommandlockouts
	% after \documentclass
	
	% for over three affiliations, or if they all won't fit within the width
	% of the page, use this alternative format:
	% 
	%\author{\IEEEauthorblockN{Michael Shell\IEEEauthorrefmark{1},
	%Homer Simpson\IEEEauthorrefmark{2},
	%James Kirk\IEEEauthorrefmark{3}, 
	%Montgomery Scott\IEEEauthorrefmark{3} and
	%Eldon Tyrell\IEEEauthorrefmark{4}}
	%\IEEEauthorblockA{\IEEEauthorrefmark{1}School of Electrical and Computer Engineering\\
	%Georgia Institute of Technology,
	%Atlanta, Georgia 30332--0250\\ Email: see http://www.michaelshell.org/contact.html}
	%\IEEEauthorblockA{\IEEEauthorrefmark{2}Twentieth Century Fox, Springfield, USA\\
	%Email: homer@thesimpsons.com}
	%\IEEEauthorblockA{\IEEEauthorrefmark{3}Starfleet Academy, San Francisco, California 96678-2391\\
	%Telephone: (800) 555--1212, Fax: (888) 555--1212}
	%\IEEEauthorblockA{\IEEEauthorrefmark{4}Tyrell Inc., 123 Replicant Street, Los Angeles, California 90210--4321}}
	
	
	
	
	% use for special paper notices
	%\IEEEspecialpapernotice{(Invited Paper)}
	
	
	
	
	% make the title area
	\maketitle
	
	% As a general rule, do not put math, special symbols or citations
	% in the abstract
	\begin{abstract}
		VOXSpell is a spelling quiz game developed for computers which uses speech synthesis to convey the word and is specifically targeted at children between the ages of 7-10. This influenced the design of interface, the supporting material (user manual and readme file) and the game mechanics. The interface keeps kids engaged by being easier to use as they have less motor control. The supporting material; specifically, the user manual does away with large words and complex sentences so the kids can read it themselves. The mechanics engages kids in learning through gamification with points systems.
		\\
		Some of its innovative features were; homophone detection, quiz word smart selection and game extension. Homophone detection is the automatic exclusion of any words that the user might be quizzed that could sound the same as other words which can lead to quite a frustrating experience when the only to get the word to spell from the game is through speech synthesis. I wanted to ensure kids were being thoroughly tested and tailored for; any words that weren't yet attempted were more likely to be in quizzes, likewise with words that have been incorrectly spelt more times. Kids like to interact with what they're given and fiddle so I tried to make resources as extendable and swappable as possible without compromising the userability and education.
	\end{abstract}
	
	% no keywords
	
	
	
	
	% For peer review papers, you can put extra information on the cover
	% page as needed:
	% \ifCLASSOPTIONpeerreview
	% \begin{center} \bfseries EDICS Category: 3-BBND \end{center}
	% \fi
	%
	% For peerreview papers, this IEEEtran command inserts a page break and
	% creates the second title. It will be ignored for other modes.
	\IEEEpeerreviewmaketitle
	
	
	
	\section{General User Interface Design}
	\subsection{Choice of Packages}
	\subsubsection{Java}
	Being very highly used at the time of development means that it is highly documented by the Java developer community and there are many GUI frameworks to choose from.\\
	Cross-platform mobility makes Java a good candidate especially with Android's primary developer language being it, thus leaving the door open for a future mobile application. All that would be needed is for a mobile frontend with modified controllers.\\
	Java is intended to be cross-platform which leads to developers of frameworks and libraries trying to retain that; this makes for developing across all PCs much easier.
	\subsubsection{JavaFX}
	Gluon scenebuilder is a drag'n'drop GUI builder with a high level of refinement which speeds up development and reduces errors. It also presents options during creation leading to more new ideas.\\
	Swing (the predecessor) will run out of support sooner which means the application's life is extended by being created with JavaFx. It also brings new features out they may be very useful during development.\\
	JavaFX when correctly done enforces the MVC pattern because all view information is put in the ".fxml" files and the controller classes are auto-generated thus increasing the maintainability and flexibility of the interface.
	\subsubsection{Festival}
	The client uses a version of Ubuntu as their operating system and Festival is easy to install and maintain with Ubuntu as it's probably the most widely used speech synthesis program there is on Linux.\\
	\subsubsection{Iconic}
	Iconic is open source so there are no legal issues in using it.\\
	It is aesthetically pleasing while being simple, clear and inoffensive.\\
	It is comprehensive and generic so there is little concern of not having an appropriate icon.\\
	\subsection{Colours and Layout}
	The colour scheme was based on a set of calming green hues because it's a naturally calming colour which keeps kids focused on the game and avoids built up frustration with the game. Green is also neutral or positive in almost all cultures. \cite{Green:Scott-Kemmis}. The Layout was intentionally kept minimal to avoid confusing and frustrating children who have a shorter temper than adults, I also tried to follow design conventions such as grouping related items together for the same reason.\\ In addition the smallest font size is 14pt so that the children can read everything without having to strain as they are still learning to read.
	\subsection{Information Presentation}
	Icons went along with text as much as possible to present the options to the player both ways so whether they are a textual or visual person they will be able to navigate with ease and they allowed for grouping of similar buttons to convey meaning quicker.\\
	Extensive tooltips were implemented instead of a dedicated help function as a traditional help function requires reading and relating to the program which is okay for adults but for younger kids who have a shorter attention span it's impossible. It's also more intuitive for a user to let their mouse hover over something while they figure out what it is. Tooltips give a much stronger connection to the function of UI items than a manual but the manual was also included for things that can't possibly be included in tooltips.\\
	The spelling of a failed word was implemented because allowed kids to associate the auditory information with the textual.\\
	UI Items intentionally had text to further exercise the reading abilities of the children and thus helping their spelling.
	\subsection{Other Challenges}
	An insignificant portion of the population is red-green colour blind \cite{colour:blind} so a design decision was made to keep the UI mostly all green so that colour blind kids would not be affected. This required the right selection of greens to keep the UI interesting and readable.
	
	\section{Functionality}
	\subsection{Motivation}
	\begin{enumerate}
		\item Words to be quizzed were picked on what the player needed to work on the most so that they wouldn't miss out on words while they progress through the game and to target their weak words instead of hiding behind their strong words.
		\item Words that sound the same are not quizzed to the user to avoid confusion and frustration, this is a limitation of using speech to deliver the word to spell.
		\item Game extension primarily through adding more words provided by the user allows them to focus on what they need to that the game no longer targets. The default list is based on the New Zealand Government�s education department for that age group but avid readers or players would soon outgrow it.
	\end{enumerate}
	\subsection{Userability Decisions}
	\begin{enumerate}
		\item The user doesn't know any better unless they read the user manual, it is entirely hidden. The only issue would be if it ended up repeating the same words forever but that is avoided by weighting in words less attempted.
		\item This is presented to the user in the manual and readme if they wish to alter it so they can be tested on those words if they wish to be but with it being hidden in the "VOXSpell" folder a novice user needn't be concerned about it.
		\item The decision to a new wordlist incorporate with the existing list was so that it wouldn't have to be reloaded on every start-up of the game and also allows for all the features available to the default wordlist such as stats and level progression. This was trading off with robustness as it's hard to control what the user adds; to combat this instruction was given in the user manual.
	\end{enumerate}
	
	\section{Code Design and Development}
	\subsection{Software Design}
	\subsubsection{Language Choice}
	Java may not have been the best language as some others allow greater cross platform abilities by using web technology and it is not as good as them. However, the developers already knew Java so development cost was saved. It also offers an extensive set of inbuilt frameworks and libraries such as JavaFX, MediaPlayer and ProcessBuilder which saves time hunting libraries and normally guarantees they are maintained and of high quality.
	\subsubsection{Libraries Used}
	Festival is the primary library used not included in Java and it was picked because there it's the most cost-effective solution for Linux as it's free. Mac OSX's in built "say" command is much nicer; it manages to pronounce every word and is much easier to understand because it sounds nicer.
	\subsection{Development Process}
	The development process used was XP (eXtreme Programming) but a modified form for a development team of one. The schedule was semi-relaxed to meet with the Client's times. It was quite effective as the clients further refined their specification as time went on and weekly meetings helped nurture this. It was also an issue because it caused unpredictable workload due to it being dependant on what the clients felt like.
	\subsection{Innovation}
	The primary implementation innovation was that of a ResourceLoader class. Java requires a resource to be in a subdirectory of a class to be accessible which leads to messy packaging. By having resources in a separate package mess was organised and with ResourceLoader implementing the singleton pattern it made resources accessible to any class which needed to load it. Duplicate code also fell as the code to get a resource only needed to be in ResourceLoader.
	\subsection{Shortcut keys}
	Using the arrow keys and space to navigate the UI was implemented because kids have better control with a keyboard and because expert users will prefer it due to being faster. Space instead of enter as a key is much larger and allows the hand to stay on the home row of the keyboard better. The arrow keys engage the kids on a game level. Not much work needed to be done and JavaFX did most of the work.
	
	\section{Evaluation and Testing}
	\subsection{Individual Results}
	Unit tests were written for particularly hard or complex classes and methods but for the most part integration revealed the large issues and was quite time effective as the application is quite small with one developer who knows the entire codebase. The mindset of a child was adopted because they think outside of the box and push the boundaries of what they are supposed to do which is very effective for finding issues and it covers most of the things they being the target audience will do. A lot of evaluation was based on the developer's experience in the field of gaming and from there what would make a fun game was tested and implemented.
	\subsection{Peer Results}
	\begin{itemize}
		\item Background music button: A mute button was added to the main menu and it no longer plays on anything else to avoid clashing.
		\item Back button in quizzes: Added but saves progress thus far as the user did attempt to spell those words.
		\item Explanation of statistics: Included in the user manual.
		\item Voice change while playing: Decided against due to cluttering the UI.
		\item Plain main menu: Intentional to avoid UI clutter and easy navigation for children due to their lesser motor skills.
		\item Custom word list: Added in the "Custom" mode
		\item Level difficulty feedback: Done in a different way with the words in it being previewed (with speech) so the user knows the difficulty beforehand.
		\item Start button for quiz: Not implemented as user can replay the word at no penalty so it would only serve to annoy power users.
		\item Repeating voice message: This is a feature of the player pressing buttons too many times and will stop, kids will find this comical so it was decided not to remove.
		\item All words shown in stats: Intentional to keep UI consistency.
		\item New method of displaying stats: Implemented stats as a table which is a much more natural way of reading.
		\item Disabling submit button: Not implemented to show to markers that my UI doesn't freeze and if the player is experienced enough they might know the word from the beginning sound. 
		\item Quiet music: Intentionally not loud so as not annoy caregivers and it is background not foreground music.
		\item Hidden save game file: Implemented but not fully so power users can do things like back up save games while avoiding novice users from accidentally deleting.
		\item Non-alphabet characters: Implemented, non-alphabet characters stop the player from submitting and children have less motor control than adults.
		\item Reset method: Intentionally this way as it restores the game to as if it is started for the first time.
		\item Review-mode bug: Not applicable as the mode was removed.
		\item Voices not working bug: Not reproducible, it was tested on UG4 with all three voices.
		\item Music customisation: Implemented, if the readme or user manual is read.
		\item More gamification: Implemented, a new experience system was introduced.
		\item File searching: Implemented under "Custom" mode.
	\end{itemize}
	
	\section{Future Work}
	If any future work is to be done the focal point should be on homophones as excluding them from the word list doesn't give the player the chance to learn the word through the game. Possible solutions include giving a dictionary definition which can be a problem for words with more than one definition or ones that have a poor definition. Another possible solution is to retrieve sentences using the word but once again curating for a good sentence is difficult with expandable word lists. The user could provide the definitions or sentences but the time involved would likely deter them. Probably the best option would be to get the speech synthesis to say the word in a specific way that doesn't sound the same but this also has the issue of English accents and quality of computer speakers.
	
	\section{Conclusion}
	VOXSpell has innovative features which engage the target audience and is well polished but has a major flaw in curating user generated content and homophone detection. This further work is outside the scope of a single developer team but if VOXSpell is successful then this could be investigated.
	
	The specifications of the clients were met but they were vague and customer acceptance testing is yet to be conducted.
	
	% trigger a \newpage just before the given reference
	% number - used to balance the columns on the last page
	% adjust value as needed - may need to be readjusted if
	% the document is modified later
	%\IEEEtriggeratref{8}
	% The "triggered" command can be changed if desired:
	%\IEEEtriggercmd{\enlargethispage{-5in}}
	
	% references section
	
	% can use a bibliography generated by BibTeX as a .bbl file
	% BibTeX documentation can be easily obtained at:
	% http://mirror.ctan.org/biblio/bibtex/contrib/doc/
	% The IEEEtran BibTeX style support page is at:
	% http://www.michaelshell.org/tex/ieeetran/bibtex/
	%\bibliographystyle{IEEEtran}
	% argument is your BibTeX string definitions and bibliography database(s)
	%\bibliography{IEEEabrv,../bib/paper}
	%
	% <OR> manually copy in the resultant .bbl file
	% set second argument of \begin to the number of references
	% (used to reserve space for the reference number labels box)
	\begin{thebibliography}{1}
		\bibitem{Green:Scott-Kemmis}
		J. Scott-Kemmis. (2016, 09 28). The Color Green [Online]. Available: http://www.empower-yourself-with-color-psychology.com/color-green.html
		\bibitem{colour:blind}
		Colour Blind Awareness. (2016, 10 23). Types of Colour Blindness [Online]. Available: http://www.colourblindawareness.org/colour-blindness/types-of-colour-blindness/
	\end{thebibliography}
	
	
	
	
	% that's all folks
\end{document}


